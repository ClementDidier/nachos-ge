\hypertarget{directory_8cc}{}\subsection{Référence du fichier filesys/directory.cc}
\label{directory_8cc}\index{filesys/directory.\+cc@{filesys/directory.\+cc}}


The directory is a table of fixed length entries; each entry represents a single file, and contains the file name, and the location of the file header on disk. The fixed size of each directory entry means that we have the restriction of a fixed maximum size for file names.  


{\ttfamily \#include \char`\"{}copyright.\+h\char`\"{}}\newline
{\ttfamily \#include \char`\"{}utility.\+h\char`\"{}}\newline
{\ttfamily \#include \char`\"{}filehdr.\+h\char`\"{}}\newline
{\ttfamily \#include \char`\"{}directory.\+h\char`\"{}}\newline


\subsubsection{Description détaillée}
The directory is a table of fixed length entries; each entry represents a single file, and contains the file name, and the location of the file header on disk. The fixed size of each directory entry means that we have the restriction of a fixed maximum size for file names. 

The constructor initializes an empty directory of a certain size; we use Read\+From/\+Write\+Back to fetch the contents of the directory from disk, and to write back any modifications back to disk.

Also, this implementation has the restriction that the size of the directory cannot expand. In other words, once all the entries in the directory are used, no more files can be created. Fixing this is one of the parts to the assignment. 