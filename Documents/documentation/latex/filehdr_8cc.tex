\hypertarget{filehdr_8cc}{}\subsection{Référence du fichier filesys/filehdr.cc}
\label{filehdr_8cc}\index{filesys/filehdr.\+cc@{filesys/filehdr.\+cc}}


The file header is used to locate where on disk the file\textquotesingle{}s data is stored. We implement this as a fixed size table of pointers -- each entry in the table points to the disk sector containing that portion of the file data (in other words, there are no indirect or doubly indirect blocks). The table size is chosen so that the file header will be just big enough to fit in one disk sector, Unlike in a real system, we do not keep track of file permissions, ownership, last modification date, etc., in the file header.  


{\ttfamily \#include \char`\"{}copyright.\+h\char`\"{}}\newline
{\ttfamily \#include \char`\"{}system.\+h\char`\"{}}\newline
{\ttfamily \#include \char`\"{}filehdr.\+h\char`\"{}}\newline


\subsubsection{Description détaillée}
The file header is used to locate where on disk the file\textquotesingle{}s data is stored. We implement this as a fixed size table of pointers -- each entry in the table points to the disk sector containing that portion of the file data (in other words, there are no indirect or doubly indirect blocks). The table size is chosen so that the file header will be just big enough to fit in one disk sector, Unlike in a real system, we do not keep track of file permissions, ownership, last modification date, etc., in the file header. 

A file header can be initialized in two ways\+: for a new file, by modifying the in-\/memory data structure to point to the newly allocated data blocks for a file already on disk, by reading the file header from disk 